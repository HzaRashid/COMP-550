\documentclass[12pt]{article}
\usepackage{amsmath}
\usepackage{graphicx}
\usepackage{enumitem}
\usepackage{geometry}
\geometry{a4paper, margin=0.4in}
\usepackage{titling}
\setlength{\droptitle}{-2.5cm}
\title{ }
\author{ Reading Assignment 4 \\ Hamza Rashid, 260971031 \\ COMP 550, Fall 2024}
\date{}
\begin{document}
\maketitle

\vspace{-5ex}
The paper, \textit{``Gender Bias in Coreference Resolution"} 
(Rudinger et al., 2018) proposes dataset schemas for measuring 
gender bias in Coreference Resolution Systems (CRS). To this end, 
the authors focus on gender bias with respect to occupations, 
evaluating the accuracy of Rule-based, Statisical, and Neural 
Coreference Systems in resolving a pronoun (male, female, or neutral) 
to a coreferent antecedent that is either an occupation or a participant. 
They constructed a challenge dataset, Winogender schemas, 
in the style of Winograd schemas, wherein a pronoun must be resolved to one of two previously mentioned 
entities in a sentence. The authors followed good practice 
by validating their hand-crafted dataset on Amazon's Mechanical Turk (MTurk) with 10-way
redunancy, with 94.9\% of responses agreeing with their intended answers. This shows that
the authors designed test sentences where correct pronoun resolution is not a function of gender. However,
they do not report on their MTurk workers' approval ratings or method for selecting them 
(e.g., qualification tests). With the Winogender schemas in hand, an unbiased model is expected to
exhibit no sensitivity to pronoun gender in its accuracy, and to resolve a male or female 
pronoun to an occupation or participant with equal likelihood. 
% Furthermore, their proposed dataset is small and biased towards North American labor statistics,
% consisting of 720 sentences covering 60 occupations with corresponding gender percentages
% available from the U.S. Bureau of Labor Statistics (BLS). 

% There is no discussion of how their evaluation dataset 
% impacts their analysis depending on the national origin of the CRS models, or how the training method and national origin 
% of training corpora for the CRS models impacts their gender bias in coreference resolution. For example,
% in Burkina Faso, the majority of managerial positions are held by women, and thus a model that is
% trained or constructed with data from this nation might exhibit opposite behaviour than what was observed in the paper regarding 
% managerial positions in the United States.

To construct the dataset, the authors 
used a list of 60 one-word occupations obtained from Caliskan et al. (2017), 
with corresponding gender percentages
available from the U.S. Bureau of Labor Statistics. For each occupation, there are two sentence 
similar templates: one in which the pronoun is
coreferent with the occupation, and one in which
it is coreferent with the participant.
For each sentence template, there are two instantiations for the participant referent (a specific
participant, e.g., “the passenger,” and a generic
paricipant, “someone.”). Thus, the resulting evaluation set contains 720 sentences: 60 occupations × 2 sentence templates per
occupation × 2 participants × 3 pronoun genders.

A key observation is that when each CRS's predictions diverge based
on pronoun gender, they do so in ways that reinforce and magnify real-world 
occupational gender disparities. As shown in figure 4 of the paper, the systems' gender
preferences for occupations correlate with BLS
and the gender statistics from text (Bergsma and Lin, 2006), which these systems access directly. 
A notable case is “gotcha” sentences in which pronoun gender does not match the occupation's 
majority gender (BLS) if the occupation is the correct answer; all models perfomed worse in this scenario. 
The paper discussed a potential case of bias amplification involving the occupation manager: 
38.5\% female in the U.S. according to BLS, and mentions of
“manager” in the B\&L resource are only 5.18\%, yet no managers 
were predicted to be female by any of the coreference systems.
The mechanism in which dataset bias could be amplified at the system level depends on the system;
a Rule-based system is susceptible to the biases of the human designer and the resulting hand-crafted rules,
a Statistical system is vulnerable to the biases of a feature function associating an occupation with a gender,
and a Neural system's pretrained embeddings is prone to capturing latent correlations between occupations and genders
due to the biases in its pre-training data. In modern, Neural-based systems, system bias can
lead to further amplification in society as the market for consumer chat-based LLM's continues to grow. For example,
the integration of Gemini in Google search is prone to gender bias in queries such as "most impactful computer scientists", 
where the contributions of Ada Lovelace are likely to be overlooked in comparison with Alan Turing.


% Furthermore, their proposed dataset is small and biased towards North American labor statistics,
% consisting of 720 sentences covering 60 occupations with corresponding gender percentages
% available from the U.S. Bureau of Labor Statistics (BLS). There is no discussion of how their evaluation dataset 
% impacts their analysis depending on the national origin of the CRS models, or how the training method and national origin 
% of training corpora for the CRS models impacts their gender bias in coreference resolution. For example,
% in Burkina Faso, the majority of managerial positions are held by women, and thus a model that is
% trained or constructed with data from this nation might exhibit opposite behaviour than what was observed in the paper regarding 
% managerial positions in the United States.

\section*{References} 
\begin{itemize}
    \item Rachel Rudinger, Jason Naradowsky, Brian Leonard, and Benjamin Van Durme. 2018. Gender Bias in Coreference Resolution. In \textit{Proceedings of the 2018 Conference of the North American Chapter of the Association for Computational Linguistics: Human Language Technologies, Volume 2 (Short Papers)}, pages 8–14, New Orleans, Louisiana. Association for Computational Linguistics.
    \item Aylin Caliskan, Joanna J. Bryson, and Arvind Narayanan. 2017. In \textit{Semantics derived automatically from language corpora contain human-like biases. Science}, 356(6334):183–186.
    \item Shane Bergsma and Dekang Lin. 2006. Bootstrapping
    path-based pronoun resolution. In \textit{Proceedings of the 21st International Conference on Computational Linguistics and 44th Annual Meeting of the Association for Computational Linguistics}, pages 33–40, Sydney, Australia. Association for Computational Linguistics.
    \item https://rshiny.ilo.org/dataexplorer39/?lang=en\&id=SDG\_T552\_NOC\_RT\_A
\end{itemize}
\end{document}
