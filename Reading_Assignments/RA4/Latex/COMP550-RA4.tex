\documentclass[11pt]{article}
\usepackage{amsmath}
\usepackage{graphicx}
\usepackage{enumitem}
\usepackage{geometry}
\geometry{a4paper, margin=0.35in}
\usepackage{titling}
\setlength{\droptitle}{-3.5cm}
\title{ }
\author{ Reading Assignment 4 \\ Hamza Rashid, 260971031 \\ COMP 550, Fall 2024}
\date{}
\begin{document}
\maketitle

\vspace{-4ex}
The paper, \textit{``Gender Bias in Coreference Resolution"} 
(Rudinger et al., 2018) proposes dataset schemas for measuring 
gender bias in Coreference Resolution Systems (CRS). To this end, 
the authors focus on gender bias with respect to occupations, 
evaluating the accuracy of Rule-based, Statisical, and Neural 
Coreference Systems in resolving a pronoun (male, female, or neutral) 
to a coreferent antecedent that is either an occupation or a participant. 
They constructed a challenge dataset, \textit{Winogender schemas}, 
in the style of \textit{Winograd schemas}, wherein a pronoun must be resolved to one of two previously mentioned 
entities in a sentence. The authors followed good practice 
by validating their hand-crafted dataset on Amazon's Mechanical Turk (MTurk) with 10-way
redunancy, with 94.9\% of responses agreeing with their intended answers. This shows that
the authors designed test sentences where correct pronoun resolution is not a function of gender. However,
they do not report on their MTurk workers' approval ratings or method for selecting them 
(e.g., qualification tests). With the Winogender schemas in hand, an unbiased model is expected to
exhibit no sensitivity to pronoun gender in its accuracy, and to resolve a male or female 
pronoun to an occupation or participant with equal likelihood. 
% Furthermore, their proposed dataset is small and biased towards North American labor statistics,
% consisting of 720 sentences covering 60 occupations with corresponding gender percentages
% available from the U.S. Bureau of Labor Statistics (BLS). 

% There is no discussion of how their evaluation dataset 
% impacts their analysis depending on the national origin of the CRS models, or how the training method and national origin 
% of training corpora for the CRS models impacts their gender bias in coreference resolution. For example,
% in Burkina Faso, the majority of managerial positions are held by women, and thus a model that is
% trained or constructed with data from this nation might exhibit opposite behaviour than what was observed in the paper regarding 
% managerial positions in the United States.

To construct the dataset, the authors 
used a list of 60 one-word occupations obtained from Caliskan et al. (2017), 
with corresponding gender percentages
available from the U.S. Bureau of Labor Statistics (BLS). For each occupation, there are two similar
sentence templates: one in which the pronoun is
coreferent with the occupation, and one in which
it is coreferent with the participant.
For each sentence template, there are two instantiations for the participant referent (a specific
participant, e.g., “the passenger,” and a generic
paricipant, “someone.”). Thus, the resulting evaluation set contains 720 sentences: 60 occupations × 2 sentence templates per
occupation × 2 participants × 3 pronoun genders.

A key observation is that when each CRS's predictions diverge based
on pronoun gender, they do so in ways that reinforce and magnify real-world 
occupational gender disparities. As shown in figure 4 of the paper, the systems' gender
preferences for occupations correlate with BLS
and the gender statistics from text (Bergsma and Lin, 2006), which these systems access directly. 
A notable case is “gotcha” sentences in which pronoun gender does not match the occupation's 
majority gender (BLS) if the occupation is the correct answer; all models perfomed worse in this scenario. 
The paper discussed a potential case of bias amplification involving the occupation manager: 
38.5\% female in the U.S. according to BLS, and mentions of
“manager” in the B\&L resource are only 5.18\%, yet no managers 
were predicted to be female by any of the coreference systems.
The mechanism in which dataset bias could be amplified at the system level depends on the system;
a Rule-based system is susceptible to the biases of the expert-knowledge of its human designer,
a Statistical system is vulnerable to the biases of a feature function associating an occupation with a gender,
and a Neural system's pre-trained embeddings is prone to encoding latent biases from its pre-training data. In all three cases, gender bias is introduced into the system as an unintended consequence of improving
its performance. The resulting system-level bias can
lead to further amplification in society as human-ai interaction continues to grow, causing a cycle of bias. For example,
the integration of Gemini in Google search is prone to gender bias in queries such as "most impactful computer scientists", 
where the contributions of Ada Lovelace are likely to be overlooked in comparison with Alan Turing.

The paper demonstrates that the preferential behavior in pronoun-occupation resolution exhibited 
by all the systems correlates both with real world employment statistics and the text statistics
that these systems use. However, they note the limitations of
{Winogender schemas}, viewing them as having high positive predictive value and low negative predictive
value. This follows from their dataset's focus on occupation
bias; the models may be good at coreference resolution in this setting, but exhibit gender
bias in different topics. The Winogender schemas
revealed varying degrees of gender bias in all three
systems. In particular, 68\% of male-female minimal pair 
test sentences are resolved differently by the Rule-based 
system; 28\% for Statistical; and 13\% for Neural. And overall, male pronouns were more likely to
be resolved to the occupation antecedent than female or neutral pronouns across all systems.

In conclusion, the paper presents precise schemas for measuring the presence of gender bias in a CRS. Their dataset
has gone through rigorous validation through crowdsourcing, and they use appropriate data (BLS and B\&L) 
to compare these systems' biases with; they are all of North American origin. The authors do no explore or inquire the generalizability of these results across more models, and there is no discussion 
on the importance of using an evaluation dataset whose national origins are the same as that of the models being evaluated. 
This is critical due to the varying degrees of gender bias across nations. Furthermore, there is no discussion of how the training method and national origin 
of training corpora for the CRS models impacts their gender bias in coreference resolution. In the end, the authors 
measured the presence occupational gender bias in Rule-based, Statistical, and Neural Coreference Resolution Systems successfully, but Winogender schemas
may be extended broadly to probe for other manifestations of gender bias.

% in Burkina Faso, the majority of managerial positions are held by women, and thus a model that is
% trained or constructed with data from this nation might exhibit opposite behaviour than what was observed in the paper regarding 
% managerial positions in the United States.
% of
% which presents a rich set of considerations such as: the national origins of the bias evaluation dataset,  
% their proposed dataset is small and biased towards North American labor statistics,
% and there is no discussion 

\section*{References} 
\begin{itemize}
    \item Rachel Rudinger, Jason Naradowsky, Brian Leonard, and Benjamin Van Durme. 2018. Gender Bias in Coreference Resolution. In \textit{Proceedings of the 2018 Conference of the North American Chapter of the Association for Computational Linguistics: Human Language Technologies, Volume 2 (Short Papers)}, pages 8–14, New Orleans, Louisiana. Association for Computational Linguistics.
    \item Aylin Caliskan, Joanna J. Bryson, and Arvind Narayanan. 2017. In \textit{Semantics derived automatically from language corpora contain human-like biases. Science}, 356(6334):183–186.
    \item Shane Bergsma and Dekang Lin. 2006. Bootstrapping
    path-based pronoun resolution. In \textit{Proceedings of the 21st International Conference on Computational Linguistics and 44th Annual Meeting of the Association for Computational Linguistics}, pages 33–40, Sydney, Australia. Association for Computational Linguistics.
    \item https://rshiny.ilo.org/dataexplorer39/?lang=en\&id=SDG\_T552\_NOC\_RT\_A
\end{itemize}
\end{document}
