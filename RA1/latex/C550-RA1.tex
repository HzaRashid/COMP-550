\documentclass[11pt]{article}
\usepackage{amsmath}
\usepackage{graphicx}
\usepackage{enumitem}
\usepackage{geometry}
\geometry{a4paper, margin=0.5in}
\title{Reading Assignment 1}
\author{Hamza Rashid \\ COMP 550, Fall 2024}
\date{}
\begin{document}
\maketitle
% \section*{Introduction}
% - Briefly introduce the paper "Neural Architectures for Named Entity Recognition" by Lample et al. (2016).
% - State that the paper presents two neural architectures for NER: bidirectional LSTM-CRF and transition-based chunking using stack-LSTMs.
% - Mention the main objective: developing language-agnostic NER systems without reliance on hand-crafted features or gazetteers.


The paper "Neural Architectures for 
Named Entity Recognition" by Lample et al. (2016)., 
addressed the reliance of state-of-the-art 
Named Entity Recognition (NER) systems on 
hand-crafted  features and domain-specific knowledge, 
a common approach at the time due to the small, 
supervised training corpora that was available. 
The authors proposed two neural 
architectures designed to generalize 
under the limited training corpora: a 
bidirectional-LSTM encoder pipelining input sequences
to a CRF layer, and, a greedy, transition-based 
chunking algorithm utilizing a Stack-LSTM to manage states. 
A key component of their methodology was
the formation of character-sensitive word 
embeddings, to capture orthographical and morphological 
details. While the proposed models acheived
state-of-the-art performance, some aspects
of the methodology limit the scope for generalization. 
We continue in more detail.

% - Describe the key approaches used:
%   - Bidirectional LSTM-CRF model.
%   - Stack-LSTM-based transition model.
% - Discuss the use of character-based word representations combined with unsupervised pre-trained embeddings to improve generalization.
% - Mention the experiments conducted on English, Dutch, German, and Spanish datasets, and their state-of-the-art results.

The Bi-LSTM encoder consists of a forward and backward 
LSTM, reading the input sequence in those orders, respectively.
The resulting left and right representations are then concatenated
into a single vector, the word-in-context representation.
The bi-directional architecture is critical 
for making informative encodings, 
as the left and right encodings 
may be insufficient, or capture the wrong \textit{context},
if considered in isolation
(homographical richness is an intuitive case). 
Furthermore, the CRF layer utilizes the formulation
discussed in class, where the feature sum is taken 
over the state-transition and observation-emission (given by the Bi-LSTM) scores.
The linear-chain CRF's ability to capture the bi-directional
relationships between output labels makes it effective in addressing 
two problems that are closely related to classification accuracy: 
the strong dependence between output labels (inherent to the NER task), 
and, the formal grammar induced by the \texttt{IOBES (Inside, Outside, Beginning, End, Single)}
tagging scheme utilized by the authors (e.g., \texttt{I-PER} cannot follow \texttt{B-LOC}).

% The forward LSTM captures the context to the left of each word, while the backward 
% captures the context to the right. The forward and backward context for each token
% are concatenated to capture the entire context.

describe transition-based chunking model



%discuss input embedding (and how it handles oov):
In desigining the input word embeddings, 
the authors found character-level 
treatment crucial for identifying 
orthographic or morphological evidence 
that something is a name (or
not a name). This was acheieved by 
initializing random embeddings 
for each character, then feeding
each token as a character sequence into 
a Bi-LSTM – with the final
embedding given by the concatenation 
of the left and right contexts. 
This embedding is then concatenated
with pretrained word embeddings
from a large corpus. During testing, 
words without an embedding in the
lookup table are mapped to an \texttt{UNK} 
embedding, and singletons
with the \texttt{UNK}  embedding are 
assigned a probability of $0.5$.
This approach 
has been found useful in 
handling \texttt{OOV} items, as the embedding
nonetheless captures character-level details.
This differs from techniques shown in class,
which deal with \texttt{OOV} items either
at the token level (\texttt{UNK} mapping, but no character-level embedding),
or at the hyperparameter-tuning level 
(e.g., the Naive Bayes smoothing parameter).
and, embeddings learned from a large corpus 
that are sensitive to word order. 


discuss results (and advantages/disadvantages of using external gazetteers)
% MENTION DROPOUT!!!!

\section*{Strengths and Limitations}
The researchers acheived the original task, 
- Strengths:
  - Language-independent, does not rely on hand-crafted features or gazetteers.
  - Achieves state-of-the-art results in multiple languages.
  - Effective use of character-based and word embeddings to handle morphology.
- Limitations:
  - The transition-based chunking model is more dependent on character-based information compared to the LSTM-CRF.
  - Greedy action selection in the Stack-LSTM model can lead to suboptimal results.
  The paper includes a detailed outline
  of the methodologies, and provides strong 
  justifications for its preprocessing decisions, 
  particularly at the input layer (arguing that LSTM's 
  are an a priori better function class for modeling 
  the relationship between words and their characters, 
  as they take into account position-variant features) .
\section*{Bidirectional LSTM-CRF Architecture (Figure 1)}
- Describe the key components:
  - Bidirectional LSTM: Encodes contextual information from both left and right contexts.
  - Conditional Random Field (CRF): Models dependencies between tags to produce globally optimal sequences.
- Explain the importance of these components:
  - Bidirectional LSTM captures comprehensive context for each word.
  - CRF layer ensures valid and coherent tag sequences.

\section*{Handling of OOV Items}
- Describe how the proposed method addresses out-of-vocabulary (OOV) words:
  - Uses character-level embeddings generated by a bidirectional LSTM to represent words based on their characters.
  - Incorporates pre-trained embeddings to handle unseen words by mapping them to a common UNK embedding during training.
- Compare to class discussions:
  - Similar to character-level models we discussed, which also leverage character features to address OOV problems.
  - Pre-trained embeddings are akin to word2vec embeddings we discussed for capturing distributional semantics.

\section*{Use of Gazetteers (Table 1)}
- Discuss methods incorporating gazetteers to improve NER performance:
  - Gazetteers can provide explicit, domain-specific named entity information, helping models generalize better.
  - Advantages: Improves recognition accuracy for specific entity types, especially in low-resource settings.
  - Disadvantages: Dependence on domain-specific resources reduces language independence and increases cost for new domains or languages.

\section*{Conclusion}
- Summarize the overall contribution of the paper.
- Highlight the effectiveness of neural architectures for NER without relying on language-specific features.

\end{document}