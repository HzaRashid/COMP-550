\documentclass[11pt]{article}
\usepackage{amsmath}
\usepackage{graphicx}
\usepackage{enumitem}
\usepackage{geometry}
\geometry{a4paper, margin=1in}
\title{Reading Assignment 1: COMP 550, Fall 2024}
\author{Hamza Rashid}
\date{}
\begin{document}
\maketitle
\section*{Introduction}
% - Briefly introduce the paper "Neural Architectures for Named Entity Recognition" by Lample et al. (2016).
% - State that the paper presents two neural architectures for NER: bidirectional LSTM-CRF and transition-based chunking using stack-LSTMs.
% - Mention the main objective: developing language-agnostic NER systems without reliance on hand-crafted features or gazetteers.


The paper "Neural Architectures for 
Named Entity Recognition" by Lample et al. (2016)., 
addressed the reliance of state-of-the-art 
Named Entity Recognition (NER) systems on 
hand-crafted  features and domain-specific knowledge, 
a common approach at the time due to the small, 
supervised training corpora that was available. 
The authors proposed two LSTM-based neural 
architectures designed to generalize without 
relying on external resources. Key 
components of their methodology included: 
IOBES tagging scheme (Inside, Outside, Beginning, 
End, Singleton), and forming character-sensitive word 
embeddings to capture orthographical and morphological 
details. While the proposed models acheive 
state-of-the-art performance, some aspects
of the methodology limit the scope for generalization. 
We delve into further detail on these matters.

% evidence that
% a token, or sequential group of tokens, is a name.
% the dependence of classes on orthographic evidence,
% While the authors
% sucessfully acheive state-of-the-art performance,
% their methodology is also limited
% Shared at the input layer of these models 
% are embedding decisions aimed to capture two intuitions: 
% (i) reasoning jointly over tagging decisions as names 
% can span multiple tokens, 
% and (ii)


\section*{Paper Content Overview}

% - Describe the key approaches used:
%   - Bidirectional LSTM-CRF model.
%   - Stack-LSTM-based transition model.
% - Discuss the use of character-based word representations combined with unsupervised pre-trained embeddings to improve generalization.
% - Mention the experiments conducted on English, Dutch, German, and Spanish datasets, and their state-of-the-art results.

The proposed neural architectures are (i) a 
bidirectional LSTM supplemented with a CRF layer, and (ii)
a greedy chunking algorithm utilizing a Stack-LSTM 
(supports stack operations and embeddings for stack objects). 
The main components of the bidirectional LSTM-CRF architecture,
depicted in Figure 1: word embeddings, Bi-LSTM encoder, and CRF layer.
The word embeddings


% \section*{Strengths and Limitations}
% The researchers acheived the original task, 
% - Strengths:
%   - Language-independent, does not rely on hand-crafted features or gazetteers.
%   - Achieves state-of-the-art results in multiple languages.
%   - Effective use of character-based and word embeddings to handle morphology.
% - Limitations:
%   - The transition-based chunking model is more dependent on character-based information compared to the LSTM-CRF.
%   - Greedy action selection in the Stack-LSTM model can lead to suboptimal results.
%   The paper includes a detailed outline
%   of the methodologies, and provides strong 
%   justifications for its preprocessing decisions, 
%   particularly at the input layer (arguing that LSTM's 
%   are an a priori better function class for modeling 
%   the relationship between words and their characters, 
%   as they take into account position-variant features) .
% \section*{Bidirectional LSTM-CRF Architecture (Figure 1)}
% - Describe the key components:
%   - Bidirectional LSTM: Encodes contextual information from both left and right contexts.
%   - Conditional Random Field (CRF): Models dependencies between tags to produce globally optimal sequences.
% - Explain the importance of these components:
%   - Bidirectional LSTM captures comprehensive context for each word.
%   - CRF layer ensures valid and coherent tag sequences.

% \section*{Handling of OOV Items}
% - Describe how the proposed method addresses out-of-vocabulary (OOV) words:
%   - Uses character-level embeddings generated by a bidirectional LSTM to represent words based on their characters.
%   - Incorporates pre-trained embeddings to handle unseen words by mapping them to a common UNK embedding during training.
% - Compare to class discussions:
%   - Similar to character-level models we discussed, which also leverage character features to address OOV problems.
%   - Pre-trained embeddings are akin to word2vec embeddings we discussed for capturing distributional semantics.

% \section*{Use of Gazetteers (Table 1)}
% - Discuss methods incorporating gazetteers to improve NER performance:
%   - Gazetteers can provide explicit, domain-specific named entity information, helping models generalize better.
%   - Advantages: Improves recognition accuracy for specific entity types, especially in low-resource settings.
%   - Disadvantages: Dependence on domain-specific resources reduces language independence and increases cost for new domains or languages.

% \section*{Conclusion}
% - Summarize the overall contribution of the paper.
% - Highlight the effectiveness of neural architectures for NER without relying on language-specific features.

\end{document}